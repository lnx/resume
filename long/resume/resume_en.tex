%% start of file `template-zh.tex'.
%% Copyright 2006-2013 Xavier Danaux (xdanaux@gmail.com).
%
% This work may be distributed and/or modified under the
% conditions of the LaTeX Project Public License version 1.3c,
% available at http://www.latex-project.org/lppl/.


\documentclass[11pt,a4paper,sans]{moderncv}   % possible options include font size ('10pt', '11pt' and '12pt'), paper size ('a4paper', 'letterpaper', 'a5paper', 'legalpaper', 'executivepaper' and 'landscape') and font family ('sans' and 'roman')

% moderncv 主题
\moderncvstyle{classic}                        % 选项参数是 ‘casual', ‘classic', ‘oldstyle' 和 'banking'
\moderncvcolor{blue}                          % 选项参数是 ‘blue' (默认)、‘orange'、‘green'、‘red'、‘purple' 和 ‘grey'
%\nopagenumbers{}                             % 消除注释以取消自动页码生成功能

% 字符编码
\usepackage[utf8]{inputenc}                   % 替换你正在使用的编码
\usepackage{CJKutf8}

% 调整页面出血
\usepackage[scale=0.75]{geometry}
%\setlength{\hintscolumnwidth}{3cm}           % 如果你希望改变日期栏的宽度

% 个人信息
\name{Long}{Zhao}
\title{Beihang University}                     % 可选项、如不需要可删除本行
% \address{No.38 Xueyuan Rd., Haidian District}{Beijing, China, 100191}            % 可选项、如不需要可删除本行
\phone[mobile]{+86-18601264124}              % 可选项、如不需要可删除本行
% \phone[fixed]{+2~(345)~678~901}               % 可选项、如不需要可删除本行
% \phone[fax]{+3~(456)~789~012}                 % 可选项、如不需要可删除本行
\email{buaastorm@gmail.com}                    % 可选项、如不需要可删除本行
% \homepage{blog.codinghonor.com}                  % 可选项、如不需要可删除本行
% \extrainfo{附加信息 (可选项)}                 % 可选项、如不需要可删除本行
% \photo[64pt][0.4pt]{photo}                  % ‘64pt'是图片必须压缩至的高度、‘0.4pt‘是图片边框的宽度 (如不需要可调节至0pt)、'picture‘ 是图片文件的名字;可选项、如不需要可删除本行
% \quote{引言(可选项)}                          % 可选项、如不需要可删除本行

% 显示索引号;仅用于在简历中使用了引言
%\makeatletter
%\renewcommand*{\bibliographyitemlabel}{\@biblabel{\arabic{enumiv}}}
%\makeatother

% 分类索引
%\usepackage{multibib}
%\newcites{book,misc}{{Books},{Others}}
%----------------------------------------------------------------------------------
%            内容
%----------------------------------------------------------------------------------
\begin{document}
\begin{CJK}{UTF8}{gbsn}                       % 详情参阅CJK文件包
\maketitle

\section{Personal Information}
\cvitem{Major}{Computer Science}
\cvitem{Mentor}{\textbf{Prof.} Shilong Ma}

\section{Education}
\cventry{2010 -- 2013}{Master}{Beihang University}{School of Computer Science}{}{Major: Mathematical Statistics, Mathematical Logic, Pattern Recognition, Uncertainty Artificial Intelligence.}  % arguments 3 to 6 are optional
\cventry{2006 -- 2010}{Bachelor}{Beihang University}{School of Computer Science}{}{Major: Discrete Mathematics, Algorithm, Operating System, Compilers Principles, Networks.}  % arguments 3 to 6 are optional


\section{Graduation Project}
\cvitem{Title}{\textbf{A Survey of the Collaborative Observation Techniques in Distributed Sensor Networks}}
\cvitem{Brief}{\small Designed a dynamic and decentralized sensor network. Proposed a half-sleep scheduling mechanism. Implemented the co-observation execution process.}

\section{Work Experience}
\cventry{2013 -- 2014}{IBM}{CSTL Web Integration}{}{}{The major developer of the WebAdmin project, a web based management tool for application servers of IBM.\newline{}%
Work Contents:%
\begin{itemize}%
\item Design and develop new features
\item Solve CPS from clients
\end{itemize}}

\section{Internship Experience}
\cventry{2012 -- 2012}{Morgan Stanley}{Financial Derivatives}{}{}{Developed a monitoring system for the financial derivatives trading system, mainly based on its running log. There are two steps here: 1. data extracting from the running log; 2. monitoring visualization (Graphiti).\newline{}}
\cventry{2011 -- 2011}{Yahoo}{Advertisement}{}{}{Developed an automatic deployment toolkit for behavior targeting team's preliminary deployment, which connects a four step pipe line operation on a hadoop cluster. If something goes wrong during deploying, it will notify the developer with an email.\newline{}}

\section{Indie Developer}
\cvitem{2008 -- 2015}{Another identity of me is an independent developer. These days I am working on a project called vox. Welcome to visit my GitHub for more info.}
\cvitem{}{\url{https://github.com/lnx}}

\section{Language Skills}
\cvitem{Coding}{Java, Python}{}
\cvitem{English}{TOEFL(94), GRE(310+3.0)}{}

\section{Awards}
\cvitem{}{Second prize of fellowship of Beihang University}{}

\section{Hobbies and interests}
\cvitem{}{Reading, travelling.}{}

% \section{其他 1}
% \cvlistitem{项目 1}
% \cvlistitem{项目 2}
% \cvlistitem{项目 3}

\renewcommand{\listitemsymbol}{-}             % 改变列表符号

% \section{其他 2}
% \cvlistdoubleitem{项目 1}{项目 4}
% \cvlistdoubleitem{项目 2}{项目 5\cite{book1}}
% \cvlistdoubleitem{项目 3}{}

% 来自BibTeX文件但不使用multibib包的出版物
%\renewcommand*{\bibliographyitemlabel}{\@biblabel{\arabic{enumiv}}}% BibTeX的数字标签
\nocite{*}
\bibliographystyle{plain}
\bibliography{publications}                    % 'publications' 是BibTeX文件的文件名

% 来自BibTeX文件并使用multibib包的出版物
%\section{出版物}
%\nocitebook{book1,book2}
%\bibliographystylebook{plain}
%\bibliographybook{publications}               % 'publications' 是BibTeX文件的文件名
%\nocitemisc{misc1,misc2,misc3}
%\bibliographystylemisc{plain}
%\bibliographymisc{publications}               % 'publications' 是BibTeX文件的文件名

\clearpage\end{CJK}
\end{document}


%% 文件结尾 `template-zh.tex'.

%% start of file `template-zh.tex'.
%% Copyright 2006-2013 Xavier Danaux (xdanaux@gmail.com).
%
% This work may be distributed and/or modified under the
% conditions of the LaTeX Project Public License version 1.3c,
% available at http://www.latex-project.org/lppl/.


\documentclass[11pt,a4paper,sans]{moderncv}   % possible options include font size ('10pt', '11pt' and '12pt'), paper size ('a4paper', 'letterpaper', 'a5paper', 'legalpaper', 'executivepaper' and 'landscape') and font family ('sans' and 'roman')

% moderncv 主题
\moderncvstyle{classic}                        % 选项参数是 ‘casual', ‘classic', ‘oldstyle' 和 'banking'
\moderncvcolor{blue}                          % 选项参数是 ‘blue' (默认)、‘orange'、‘green'、‘red'、‘purple' 和 ‘grey'
%\nopagenumbers{}                             % 消除注释以取消自动页码生成功能

% 字符编码
\usepackage[utf8]{inputenc}                   % 替换你正在使用的编码
\usepackage{CJKutf8}

% 调整页面出血
\usepackage[scale=0.9,top=1.6cm]{geometry}
%\setlength{\hintscolumnwidth}{3cm}           % 如果你希望改变日期栏的宽度

% 个人信息
\name{Long}{Zhao}
\title{Beihang University}
% \address{No.38 Xueyuan Rd., Haidian District}{Beijing, China, 100191}
\phone[mobile]{+86-18601264124}
\email{buaastorm@gmail.com}
% \homepage{blog.codinghonor.com}
\social[github][github.com/lnx]{https://www.github.com/lnx}
% \extrainfo{\githubsocialsymbol \httplink{https://www.github.com/lnx}}
% \photo[64pt][0.01pt]{photo}
% \quote{引言(可选项)}

% 显示索引号;仅用于在简历中使用了引言
%\makeatletter
%\renewcommand*{\bibliographyitemlabel}{\@biblabel{\arabic{enumiv}}}
%\makeatother

% 分类索引
%\usepackage{multibib}
%\newcites{book,misc}{{Books},{Others}}
%----------------------------------------------------------------------------------
%            内容
%----------------------------------------------------------------------------------
\begin{document}
\begin{CJK}{UTF8}{gbsn}                       % 详情参阅CJK文件包
\maketitle

\vspace*{-0.7cm}

\section{Education}
\cventry{2010 -- 2013}{Beihang University}{\small Master}{\small \emph Computer Science}{\small \emph GPA: 3.4}{Major: Mathematical Statistics, Mathematical Logic, Pattern Recognition, Uncertainty AI.}  % arguments 3 to 6 are optional
\cventry{2006 -- 2010}{Beihang University}{\small Bachelor}{\small \emph Computer Science}{\small \emph GPA: 3.6}{Major: Discrete Mathematics, Algorithm, Operating System, Compilers Principles, Networks.}  % arguments 3 to 6 are optional

\section{Employment}
\cventry{2013 -- 2014}{IBM}{\small CSTL Web Integration Team}{}{}{}

\cventry{}{\small WebAdmin}{\small \emph{a web based configuration and management software on system i for application servers}}{}{}{
\begin{itemize}%
\item Implemented the new features for supporting Apache server upgrading from 2.2 to 2.4.
\item Designed a template engine as a supplement for the original view layer; avoided some of the repetitive work.
\item Refactored part of the legacy code according to Code Complete; reduced 20\% of the code.
\item Collaborated with colleagues to solve CPS from our clients.
\end{itemize}}

\cventry{}{\small CodeSync}{\small \emph{an automatic code submission tool for CMVC}}{}{}{
\begin{itemize}%
\item Analyzed all the files in directory metadata to detect the stauts of each monitored code file.
\item Implemented a user interface by using a pure JavaScript framework Node-Webkit.
\item Shared this tool to group members; saved nearly half of their time on code submission.
\end{itemize}}

\section{Internship}

\cventry{2012 -- 2012}{Morgan Stanley}{\small Financial Derivatives Team}{}{}{}

\cventry{}{\small TradingMonitor}{\small \emph{a monitoring system for the financial derivatives trading system}}{}{}{
\begin{itemize}%
\item Created a syntactic analyzer to analyze system running log; be used to detect the system status.
\item Designed a minimal database; supported select, insert operation and file split.
\item Visualized the monitor data by using Graphiti, which was real-time and could be visited through web.
\item Practiced scrum with team members on this project.
\item Learned how to make amazing coffee.
\end{itemize}}

\cventry{2011 -- 2011}{Innovation Works}{\small Feibo Team}{}{}{}

\cventry{}{\small Feibo}{\small \emph{yet another WeChat}}{}{}{
\begin{itemize}%
\item Implemented part of the service API; released as a series of RESTful services by using Flask.
\item Designed friends recommendation module based on node similarities.
\end{itemize}}

\section{Projects}

\cventry{2014 -- 2015}{SitFish}{\small http://www.sitfish.com}{}{}{A site to connect lovers with music. Even they are at different places, they could still listen to the same music at the same time. If your lover switched the song, your player will automatically switched to the new song almost simultaneously and vice versa. Music share will never be such an easy and joyful thing.}
\cvitem{}{\small \emph{Techniques: Python, JavaScript, Flask, MongoDB, WebSocket.}}

\cventry{2010 -- 2015}{Others}{}{}{}{
\begin{itemize}%
\item PyWin: a Python version management tool on Windows, automatically find and help user to switch.
\item StockData: a stock data collector for heavy holdings.
\end{itemize}}

%\section{Thesis}
%\cvitem{Title}{\textbf{A Survey of the Collaborative Observation Techniques in Distributed Sensor Networks}}
%\cvitem{Brief}{\small Designed a dynamic and decentralized sensor network. Proposed a half-sleep scheduling mechanism. Implemented the co-observation %execution process.}

\section{Additional}
\cventry{}{\small Skills}{\small Java, Python, JavaScript, Flask, SQL(MySQL), NoSQL(MongoDB)}{}{}{}
\cventry{}{\small Awards}{\small Second prize of fellowship of Beihang University}{}{}{}
\cventry{}{\small Language}{\small TOEFL(94), GRE(313+3.0)}{}{}{}
\cventry{}{\small Self-Evaluation}{\small Collaborative, self-driven and fast-learning}{}{}{}

% \section{其他 1}
% \cvlistitem{项目 1}
% \cvlistitem{项目 2}
% \cvlistitem{项目 3}

\renewcommand{\listitemsymbol}{-}             % 改变列表符号

% \section{其他 2}
% \cvlistdoubleitem{项目 1}{项目 4}
% \cvlistdoubleitem{项目 2}{项目 5\cite{book1}}
% \cvlistdoubleitem{项目 3}{}

% 来自BibTeX文件但不使用multibib包的出版物
%\renewcommand*{\bibliographyitemlabel}{\@biblabel{\arabic{enumiv}}}% BibTeX的数字标签
\nocite{*}
\bibliographystyle{plain}
\bibliography{publications}                    % 'publications' 是BibTeX文件的文件名

% 来自BibTeX文件并使用multibib包的出版物
%\section{出版物}
%\nocitebook{book1,book2}
%\bibliographystylebook{plain}
%\bibliographybook{publications}               % 'publications' 是BibTeX文件的文件名
%\nocitemisc{misc1,misc2,misc3}
%\bibliographystylemisc{plain}
%\bibliographymisc{publications}               % 'publications' 是BibTeX文件的文件名

\clearpage\end{CJK}
\end{document}


%% 文件结尾 `template-zh.tex'.

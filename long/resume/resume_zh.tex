%% start of file `template-zh.tex'.
%% Copyright 2006-2013 Xavier Danaux (xdanaux@gmail.com).
%
% This work may be distributed and/or modified under the
% conditions of the LaTeX Project Public License version 1.3c,
% available at http://www.latex-project.org/lppl/.


\documentclass[11pt,a4paper,sans]{moderncv}   % possible options include font size ('10pt', '11pt' and '12pt'), paper size ('a4paper', 'letterpaper', 'a5paper', 'legalpaper', 'executivepaper' and 'landscape') and font family ('sans' and 'roman')

% moderncv 主题
\moderncvstyle{classic}                        % 选项参数是 ‘casual', ‘classic', ‘oldstyle' 和 'banking'
\moderncvcolor{blue}                          % 选项参数是 ‘blue' (默认)、‘orange'、‘green'、‘red'、‘purple' 和 ‘grey'
%\nopagenumbers{}                             % 消除注释以取消自动页码生成功能

% 字符编码
\usepackage[utf8]{inputenc}                   % 替换你正在使用的编码
\usepackage{CJKutf8}

% 调整页面出血
\usepackage[scale=0.9,top=1.6cm]{geometry}
%\setlength{\hintscolumnwidth}{3cm}           % 如果你希望改变日期栏的宽度

% 个人信息
\name{赵}{龙}
\title{北京航空航天大学}
% \address{No.38 Xueyuan Rd., Haidian District}{Beijing, China, 100191}
\phone[mobile]{+86-18601264124}
\email{buaastorm@gmail.com}
% \homepage{blog.codinghonor.com}
\social[github][github.com/lnx]{https://www.github.com/lnx}
% \extrainfo{\githubsocialsymbol \httplink{https://www.github.com/lnx}}
% \photo[64pt][0.01pt]{photo}
% \quote{引言(可选项)}

% 显示索引号;仅用于在简历中使用了引言
%\makeatletter
%\renewcommand*{\bibliographyitemlabel}{\@biblabel{\arabic{enumiv}}}
%\makeatother

% 分类索引
%\usepackage{multibib}
%\newcites{book,misc}{{Books},{Others}}
%----------------------------------------------------------------------------------
%            内容
%----------------------------------------------------------------------------------
\begin{document}
\begin{CJK}{UTF8}{gbsn}                       % 详情参阅CJK文件包
\maketitle

\vspace*{-1.05cm}

\section{教育背景}
\cventry{2010 -- 2013}{北京航空航天大学}{\small 硕士}{\small \emph{计算机科学与技术}}{\small \emph GPA: 3.4}{主修: 数理统计, 数理逻辑, 模式识别, 不确定性人工智能.}  % arguments 3 to 6 are optional
\cventry{2006 -- 2010}{北京航空航天大学}{\small 学士}{\small \emph{计算机科学与技术}}{\small \emph GPA: 3.6}{主修: 离散数学, 算法, 操作系统, 编译原理, 网络.}  % arguments 3 to 6 are optional

\section{工作经历}
\cventry{2013 -- 2014}{IBM}{\small CSTL Web集成组}{}{}{}

\cventry{}{\small WebAdmin}{\small \emph{基于web的配置管理i系统上应用服务器的软件}}{}{}{
\begin{itemize}%
\item 实现了用于支持Apache2.2到2.4升级的新功能.
\item 设计了一个模板引擎作为原有视图层的补充, 避免了开发中部分的重复性劳动.
\item 参考代码大全中的建议, 重构了部分项目遗留代码, 相关代码精简了约20\%.
\item 与同事合作解决来自客户的CPS.
\end{itemize}}

\cventry{}{\small CodeSync}{\small \emph{为CMVC开发的代码自动提交工具}}{}{}{
\begin{itemize}%
\item 分析metadata目录中的所有文件, 检测被监控的代码源文件的当前状态.
\item 利用一个纯JavaScript框架Node-Webkit实现了用户界面.
\item 把该工具分享给同组组员使用, 在代码提交上节省了大家约一半的时间.
\end{itemize}}

\section{实习经历}

\cventry{2012 -- 2012}{摩根士丹利}{\small 金融衍生品组}{}{}{}

\cventry{}{\small 交易系统监控}{\small \emph{一个监控金融衍生品交易系统的软件系统}}{}{}{
\begin{itemize}%
\item 实现了一个分析系统运行日志的语法分析器, 用于检测系统当前的运行状态.
\item 设计了一个精简数据库, 支持查询和插入操作,并能在记录文件过大时进行自动拆分.
\item 利用Graphiti对监控数据进行可视化,实时且可通过web访问.
\item 在该项目上与项目成员实践敏捷软件开发(scrum)的模式.
\item 学会了如何正确的使用咖啡机.
\end{itemize}}

\cventry{2011 -- 2011}{创新工场}{\small 飞波组}{}{}{}

\cventry{}{\small 飞波}{\small \emph{一款微信类似的聊天产品}}{}{}{
\begin{itemize}%
\item 完成了部分服务端API的实现, 并通过Flask将其发布为一系列RESTful的服务.
\item 基于节点相似性设计了好友推荐模块.
\end{itemize}}

\section{项目经验}

\cventry{2014 -- 2015}{坐鱼}{\small http://www.sitfish.com}{}{}{一个情侣间的音乐共享网站. 即便两人在不同的地方, 依然可以通过坐鱼(SitFish)来同时同听一首歌, 两人互为对方的主播. 旨在让音乐共享变得简单且令人愉悦.}
\cvitem{}{\small \emph{相关技术: Python, JavaScript, Flask, MongoDB, WebSocket.}}

\cventry{2010 -- 2015}{其它}{}{}{}{
\begin{itemize}%
\item PyWin: Windows上的Python版本管理工具,自动寻找系统已安装的Python版本,帮助用户一键切换.
\item StockData: 从网上抓取机构持股, 基金重仓, 社保重仓及QFII重仓的持股数据并入库为后续分析做准备.
\end{itemize}}

%\section{Thesis}
%\cvitem{Title}{\textbf{A Survey of the Collaborative Observation Techniques in Distributed Sensor Networks}}
%\cvitem{Brief}{\small Designed a dynamic and decentralized sensor network. Proposed a half-sleep scheduling mechanism. Implemented the co-observation %execution process.}

\section{补充说明}
\cventry{}{\small 技能}{\small Java, Python, JavaScript, Flask, SQL(MySQL), NoSQL(MongoDB)}{}{}{}
\cventry{}{\small 语言}{\small TOEFL(94), GRE(313+3.0)}{}{}{}
\cventry{}{\small 奖励}{\small 北航之友奖学金}{}{}{}
\cventry{}{\small 自评}{\small 合作, 自律, 敏而学.}{}{}{}

% \section{其他 1}
% \cvlistitem{项目 1}
% \cvlistitem{项目 2}
% \cvlistitem{项目 3}

\renewcommand{\listitemsymbol}{-}             % 改变列表符号

% \section{其他 2}
% \cvlistdoubleitem{项目 1}{项目 4}
% \cvlistdoubleitem{项目 2}{项目 5\cite{book1}}
% \cvlistdoubleitem{项目 3}{}

% 来自BibTeX文件但不使用multibib包的出版物
%\renewcommand*{\bibliographyitemlabel}{\@biblabel{\arabic{enumiv}}}% BibTeX的数字标签
\nocite{*}
\bibliographystyle{plain}
\bibliography{publications}                    % 'publications' 是BibTeX文件的文件名

% 来自BibTeX文件并使用multibib包的出版物
%\section{出版物}
%\nocitebook{book1,book2}
%\bibliographystylebook{plain}
%\bibliographybook{publications}               % 'publications' 是BibTeX文件的文件名
%\nocitemisc{misc1,misc2,misc3}
%\bibliographystylemisc{plain}
%\bibliographymisc{publications}               % 'publications' 是BibTeX文件的文件名

\clearpage\end{CJK}
\end{document}


%% 文件结尾 `template-zh.tex'.

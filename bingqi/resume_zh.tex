%% start of file `template-zh.tex'.
%% Copyright 2006-2013 Xavier Danaux (xdanaux@gmail.com).
%
% This work may be distributed and/or modified under the
% conditions of the LaTeX Project Public License version 1.3c,
% available at http://www.latex-project.org/lppl/.


\documentclass[11pt,a4paper,sans]{moderncv}   % possible options include font size ('10pt', '11pt' and '12pt'), paper size ('a4paper', 'letterpaper', 'a5paper', 'legalpaper', 'executivepaper' and 'landscape') and font family ('sans' and 'roman')

% moderncv 主题
\moderncvstyle{classic}                        % 选项参数是 ‘casual’, ‘classic’, ‘oldstyle’ 和 ’banking’
\moderncvcolor{blue}                          % 选项参数是 ‘blue’ (默认)、‘orange’、‘green’、‘red’、‘purple’ 和 ‘grey’
%\nopagenumbers{}                             % 消除注释以取消自动页码生成功能

% 字符编码
\usepackage[utf8]{inputenc}                   % 替换你正在使用的编码
\usepackage{CJKutf8}

% 调整页面出血
\usepackage[scale=0.75]{geometry}
%\setlength{\hintscolumnwidth}{3cm}           % 如果你希望改变日期栏的宽度

% 个人信息
\name{吕}{冰琪}
\title{北京大学医学部}                     % 可选项、如不需要可删除本行
\address{北京市海淀区学院路38号}{100191}            % 可选项、如不需要可删除本行
\phone[mobile]{+86-18601264142}              % 可选项、如不需要可删除本行
% \phone[fixed]{+2~(345)~678~901}               % 可选项、如不需要可删除本行
% \phone[fax]{+3~(456)~789~012}                 % 可选项、如不需要可删除本行
\email{lvbingqi@gmail.com}                    % 可选项、如不需要可删除本行
% \homepage{www.xialongli.com}                  % 可选项、如不需要可删除本行
% \extrainfo{附加信息 (可选项)}                 % 可选项、如不需要可删除本行
\photo[64pt][0.4pt]{photo}                  % ‘64pt’是图片必须压缩至的高度、‘0.4pt‘是图片边框的宽度 (如不需要可调节至0pt)、’picture‘ 是图片文件的名字;可选项、如不需要可删除本行
% \quote{引言(可选项)}                          % 可选项、如不需要可删除本行

% 显示索引号;仅用于在简历中使用了引言
%\makeatletter
%\renewcommand*{\bibliographyitemlabel}{\@biblabel{\arabic{enumiv}}}
%\makeatother

% 分类索引
%\usepackage{multibib}
%\newcites{book,misc}{{Books},{Others}}
%----------------------------------------------------------------------------------
%            内容
%----------------------------------------------------------------------------------
\begin{document}
\begin{CJK}{UTF8}{gbsn}                       % 详情参阅CJK文件包
\maketitle

\section{基本信息}
\cvdoubleitem{导师}{余小鸣教授}{毕业时间}{2015年7月}
\cvdoubleitem{籍贯}{陕西咸阳}{出生年月}{1989年4月}

\section{教育背景}
\cventry{2012 -- 2015}{硕士}{北京大学医学部}{公共卫生学院}{儿少卫生与妇幼保健学}{主修:流行病学、统计学、高级公共卫生学、卫生事业管理、高级儿少卫生、健康行为学、青少年行为发展与健康、学校卫生与健康促进、SPSS统计分析。}  % 第3到第6编码可留白
\cventry{2007 -- 2011}{学士}{北京中医药大学}{护理学院}{护理学}{主修:正常人体解剖学、生理学、中医学基础、病理学、药理学、护理学基础、精神病护理学、健康评估、护理心理学、内外妇儿护理学、社区护理学、急救护理学。}

\section{毕业课题}
\cvitem{题目}{\textbf{高校在校大学生健康体检意向及其影响因素研究}}
\cvitem{导师}{余小鸣}
\cvitem{简介}{\small 以计划行为理论为框架,对大学生健康体检行为意向进行探索,为提高大学生健康体检意向促进大学生健康行为的形成提供建议。}

\section{论文成果}
\cventry{1.}{16-20岁城市务工青少年心理社会能力及其影响因素分析}{中国儿童保健杂志}{}{}{\scriptsize \textbf{吕冰琪},余小鸣,高素红,张淑平,周晓梅,曾芳玲}
\cventry{2.}{城市务工青少年心理社会能力分析}{中华预防医学会第四届学术年会}{}{}{\scriptsize \textbf{吕冰琪},余小鸣,高素红,张淑平,周晓梅,曾芳玲}
\cventry{3.}{高中生健康素养评价问卷的结构框架及信效度分析}{中国学校卫生}{}{}{\scriptsize 余小鸣,郭帅军,王璐,仇元营,\textbf{吕冰琪},王嘉}
\cventry{4.}{学校卫生工作监测与评估指南}{联合国教科文组织}{}{}{\scriptsize 参与翻译}
\cventry{5.}{健康教育教学指导}{高教出版社}{}{}{\scriptsize 参与编写}

\section{科研经历}
\cventry{13.09 -- 15.05}{大学生健康体检管理办法研究}{教育部}{}{}{调查高校健康体检现况,了解高校学生及校医院院长对健康体检相关问题的认识与看法,为制定与完善大学生健康体检政策提供依据。\newline{}%
工作内容:%
\begin{itemize}%
\item 整理分析并撰写“高校健康体检相关情况调查报告”;
\item 设计编制“大学生健康体检调查问卷”,负责全国5个省市20所高校的现场联系、调研、问卷录入及报告撰写。
\end{itemize}}
\cventry{13.03 -- 15.05}{北京市十二五教育规划课题“对中小学健康素养的评价研究”}{教育部}{}{}{通过探索学生健康素养内涵,构建评价指标体系,建立评价题库,随机形成调查问卷,并进行调查验证。\newline{}%
工作内容:%
\begin{itemize}%
\item 参与中小学生健康素养评价题库的建设;
\item 中小学在校生现场调查、数据录入及调查报告撰写。
\end{itemize}}
\cventry{14.05 -- 14.11}{“乐童成长计划”学校健康促进国际合作项目}{国际儿童救助会}{}{}{了解我国不同地区健康教育开展及健康教育政策落实情况,为增强《中小学健康教育指导纲要》的进一步落实及改进《纲要》中的健康教育基本内容提供建议。\newline{}%
工作内容:%
\begin{itemize}%
\item 参与定性与定量研究工具的制定;
\item 北京地区现场调研;
\item 问卷录入。
\end{itemize}}
\cventry{13.09 -- 13.10}{全国高校校医院院长管理培训}{教育部}{}{}{为加强高校校医院管理工作,提高工作质量,并满足高校广大师生日益增长的卫生保健需求,对全国高校校医院院长进行相关培训。\newline{}%
工作内容:%
\begin{itemize}%
\item 全程参与会议筹备工作、会务及后期资料整理工作。
\end{itemize}}

\section{实习经历}
\cventry{13.12 -- 14.05}{教育部}{体卫艺司卫生处}{}{}{%
\begin{itemize}%
\item 协助“2014年全国学生体质调研”函的会签、发放,并整理2014年体质调研各省点校、经费的汇总统计;
\item 整理和统计全国各省幼儿园及中小学校健康服务管理排查情况;
\item 2013年学校突发公共卫生事件统计;
\item 人大和政协提案整理。
\end{itemize}}
\cventry{13.09 -- 13.11}{北京市朝阳区疾病预防与控制中心}{学校卫生与牙防科}{}{}{%
\begin{itemize}%
\item 参与“北京市中小学校体育课运动负荷监测与评价工作”,共监测6所学校,40节体育课,并完成数据录入及整理;
\item 撰写2013年度朝阳区疾病预防控制中心学校卫生与牙防科工作总结大会中生长发育部分报告并发言;
\item 参与“学校教室微小气候监测”,对朝阳区24所学校,48间教室的二氧化碳浓度和温度进行了监测;
\item 参加“全国爱牙日”活动策划会议、牙防宣传材料的发放等;
\item 撰写“北京市朝阳区烟草监测结果分析”报告;
\item 对朝阳区8所学校的物质环境进行监测。
\end{itemize}}
\cventry{10.07 -- 11.05}{护理学专业实践}{}{}{}{%
学习基本护理操作规范及技能:%
\begin{itemize}%
\item 中日友好医院
	\begin{itemize}%
	\item 手术室
	\item ICU
	\end{itemize}
\item 东直门医院
	\begin{itemize}%
	\item 周围血管科
	\end{itemize}
\item 朝阳医院
	\begin{itemize}%
	\item 胸外科
	\end{itemize}
\end{itemize}}

\section{获奖情况}
\cventry{2012 -- 2015}{北京大学二等奖学金}{}{}{}{}

\section{软件技能}
\cventry{1.}{SPSS}{}{}{\small 论文数据的处理及分析}{}
\cventry{2.}{EpiData}{}{}{\small 毕业课题的数据库建立及录入}{}
\cventry{3.}{Gnuplot}{}{}{\small 论文图表绘制}{}
\cventry{4.}{Office}{}{}{\small Word Excel PPT Visio均能在日常工作中熟练使用}{}
\cventry{5.}{Access}{}{}{\small 计算机二级}{}
\cventry{6.}{Mac OS}{}{}{\small 日常工作娱乐}{}

\section{语言技能}
\cvitem{英语}{四六级(500+) 托福(90+) GRE(310+3.0)}{}

\section{兴趣爱好}
\cvitem{}{旅游 烹饪 读书 看电影}{}

% \section{其他 1}
% \cvlistitem{项目 1}
% \cvlistitem{项目 2}
% \cvlistitem{项目 3}

\renewcommand{\listitemsymbol}{-}             % 改变列表符号

% \section{其他 2}
% \cvlistdoubleitem{项目 1}{项目 4}
% \cvlistdoubleitem{项目 2}{项目 5\cite{book1}}
% \cvlistdoubleitem{项目 3}{}

% 来自BibTeX文件但不使用multibib包的出版物
%\renewcommand*{\bibliographyitemlabel}{\@biblabel{\arabic{enumiv}}}% BibTeX的数字标签
\nocite{*}
\bibliographystyle{plain}
\bibliography{publications}                    % 'publications' 是BibTeX文件的文件名

% 来自BibTeX文件并使用multibib包的出版物
%\section{出版物}
%\nocitebook{book1,book2}
%\bibliographystylebook{plain}
%\bibliographybook{publications}               % 'publications' 是BibTeX文件的文件名
%\nocitemisc{misc1,misc2,misc3}
%\bibliographystylemisc{plain}
%\bibliographymisc{publications}               % 'publications' 是BibTeX文件的文件名

\clearpage\end{CJK}
\end{document}


%% 文件结尾 `template-zh.tex'.
